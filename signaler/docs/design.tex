\documentclass{article}
\usepackage[utf8]{inputenc}

\title{Signaler Design-Plan}
\author{jkaten}
\date{April 2021}

\begin{document}

\maketitle

\section{Project Summary}

A program that prints out increasing prime numbers to standard output, approximately one every second.

\section{Architecture}

\subsection{Data}

None.

\subsection{Significant Functions}

\textbf{void gen\_primes(void)}
\\[.5cm]
Creates threads for printing and generating prime numbers.
\\[.5cm]
\textbf{void reverse\_prime(int sig)}
\\[.5cm]
Thread function for reversing the order in which prime numbers are generated.
\\[.5cm]
\textbf{void skip\_prime(int sig)}
Thread function for skipping the next prime in the sequence.
\\[.5cm]
\textbf{void restart\_prime(int sig)}
Thread function for restarting the prime number.
\\[.5cm]
\textbf{void *print\_thread(void *n)}
Thread function for sleeping the and printing prime number.
\\[.5cm]
\textbf{void *work\_thread(void *n)}
Thread function to generate next prime number.
\\[.5cm]
\textbf{bool is\_prime(size\_t n)}
Function that checks if a number is prime. Code borrowed from https://www.geeksforgeeks.org/program-to-find-the-next-prime-number.
\\[.5cm]
\textbf{size\_t next\_prime(size\_t N)}
Function that gets the next prime number is sequence. Code borrowed from https://www.geeksforgeeks.org/program-to-find-the-next-prime-number.
\\[.5cm]
\textbf{int handle\_args(int argc, char **argv)}
\\[.5cm]
This function will handle command line arguments.
\\[.5cm]
\textbf{void print\_help(void)}
This function prints the usage statement.

\section{Plan}
Get prime number generated in a way that works with the requirements. Work on handling the signals SIGHUP, SIGUSR1 and SIGUSR2. Then begine working on the bonus features.


\end{document}
